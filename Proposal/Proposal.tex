\documentclass{article} % For LaTeX2e
\usepackage{nips15submit_e,times}
\usepackage{hyperref}
\usepackage{url}
%\documentstyle[nips14submit_09,times,art10]{article} % For LaTeX 2.09


\title{Recommending Functions in Spreadsheets from the Fuse Corpus}


\author{
Shaown Sarker, Matthew Neal, Nisarg Vinchhi\\
Department of Computer Science\\
North Carolina State University\\
Raleigh, NC 27606 \\
\texttt{\{ssarker,meneal,nvinchh\}@ncsu.edu}
}

% The \author macro works with any number of authors. There are two commands
% used to separate the names and addresses of multiple authors: \And and \AND.
%
% Using \And between authors leaves it to \LaTeX{} to determine where to break
% the lines. Using \AND forces a linebreak at that point. So, if \LaTeX{}
% puts 3 of 4 authors names on the first line, and the last on the second
% line, try using \AND instead of \And before the third author name.

\newcommand{\fix}{\marginpar{FIX}}
\newcommand{\new}{\marginpar{NEW}}

\nipsfinalcopy % Uncomment for camera-ready version

\begin{document}


\maketitle

% We don't want any page numbering for a single page document.
\thispagestyle{empty}

\section*{Project Objective}
Spreadsheets are the most common form of end-user programming. Although 
spreadsheets have a large array of functions built-in, spreadsheet users often 
do not exploit them to perform their tasks efficiently. To address this issue, 
in this project we investigate recommender system technologies and consider two 
distinct approaches to a collaborative filtering based function recommender 
system for spreadsheets.

Our main inspiration comes from prior research on recommending commands in 
AutoCAD [1] and Eclipse [2]. The system described in both papers utilized 
user-based and item-based collaborative filtering algorithms on a collective 
users' command usage history to recommend personalized commands given the usage 
history of an individual user as input. We intend to use the same algorithms on
Fuse\footnote{http://static.barik.net/fuse/}, the largest and most diverse 
spreadsheet corpora to date, to recommend personalized functions for an 
individual given her function usage in the form of a collection of spreadsheets.

One key difference in our approach is that unlike the prior command 
recommendation systems which had access to sequential command usage history, we
do not have similar temporal information as spreadsheet files in the corpus are
static and does not contain such function usage history. We will depend on 
function usage frequencies instead to utilize the recommender algorithms. Our
work will involve feature vector extraction from the Fuse corpus, applying the
collaborative filtering algorithms to recommend functions for the input user,
and validate the efficacy of the recommendations by cross validation.

\section*{Team Members}
Our team consists of three members: Shaown Sarker will be involved in feature
extraction and implementation of one of the algorithms, Matthew Neal will be
implementing the other algorithm, and Nisarg Vinchhi will be involved in the 
cross validation. By April 5th, we intend to have implemented both the 
algorithms and the framework required for the cross validation.

\subsubsection*{References}
\small{
[1] Matejka, J., Li, W., Grossman, T., \& Fitzmaurice, G. (2009). 
CommunityCommands: command recommendations for software applications. 
{\it{In Proceedings of the 22nd annual ACM symposium on User interface software 
and technology (pp. 193-202).}} ACM.

[2] Murphy-Hill, E., Jiresal, R., \& Murphy, G. C. (2012). 
Improving software developers' fluency by recommending development environment
commands. {\it{In Proceedings of the ACM SIGSOFT 20th International Symposium on
the Foundations of Software Engineering (p. 42).}} ACM.

\end{document}