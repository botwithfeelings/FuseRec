\documentclass{article} % For LaTeX2e
\usepackage{nips15submit_e,times}
\usepackage{hyperref}
\usepackage{url}
%\documentstyle[nips14submit_09,times,art10]{article} % For LaTeX 2.09


\title{Recommending Functions in Spreadsheets from the Fuse Corpus}


\author{
Shaown Sarker, Matthew Neal, Nisarg Vinchhi\\
Department of Computer Science\\
North Carolina State University\\
Raleigh, NC 27606 \\
\texttt{\{ssarker,meneal,nvinchh\}@ncsu.edu}
}

% The \author macro works with any number of authors. There are two commands
% used to separate the names and addresses of multiple authors: \And and \AND.
%
% Using \And between authors leaves it to \LaTeX{} to determine where to break
% the lines. Using \AND forces a linebreak at that point. So, if \LaTeX{}
% puts 3 of 4 authors names on the first line, and the last on the second
% line, try using \AND instead of \And before the third author name.

\newcommand{\fix}{\marginpar{FIX}}
\newcommand{\new}{\marginpar{NEW}}

\nipsfinalcopy % Uncomment for camera-ready version

\begin{document}


\maketitle

\begin{abstract}
Spreadsheets are the most widely used form of end-user programming. Although spreadsheets have a large array of functions built-in, the majority of spreadsheet users often do not utilize them to perform their tasks. To address this lack of function usage, we investigate recommender system technologies and consider two distinct approaches to a function recommender system for spreadsheets. In our work, we apply two variations of collaborative filtering algorithm to produce personalized function recommendations to spreadsheet users by applying these algorithms on the Fuse spreadsheet corpus. We compare the performance of the algorithms used with a baseline of most popular algorithm, also known as Linton's algorithm. In this paper, we present the feature extraction process from the spreadsheet corpus, the algorithms used and their comparative performance. Finally, we also outline the roadmap to use our findings to create an Excel plugin for increasing functional awareness.
\end{abstract}

\section{Introduction}
% The spreadsheet users are important.
Ko et al. define end-user programmers as people who are not professional software developers, but make use of tools and processes that lets them perform tasks similar to programming~\cite{ko2011state}. According to a study from 2005~\cite{scaffidi2005estimating}, nearly 23 million Americans use spreadsheets, constituting 30\% of the entire workforce. Spreadsheets are used not only for home and small businesses, but they are also very commonly used in industry. Almost 90\% of all analysts use spreadsheets to perform their calculations~\cite{winston2001executive}. Based on their numbers, spreadsheet users form the largest demographic within end-user programmers.

% There are lots of function, but people don't really use them.
Spreadsheet software come with lots of functions built-in. Consider Microsoft Excel, the most widely used spreadsheet application, which has more than 300 functions\footnote{https://support.office.com/en-us/article/Excel-functions-by-category-5f91f4e9-7b42-46d2-9bd1-63f26a86c0eb}. But there are many situations where the user neglects using them. Let us consider the case of a spreadsheet user, Titus. Titus is a fourth grade teacher in an elementary school. At the end of the school year, he wants to calculate the class grades in Excel from all the test scores entered into his spreadsheet. Instead of using the arithmetic functions in Excel, he reaches for his hand-held calculator, and uses it to calculate and enter the grade for each student. This type of usage, not taking advantage of the functions in spreadsheets, is very common with users where they lack awareness of functionality~\cite{grossman2009survey} of the end-user programming tool being used.

% Address the issue with recommender systems.
Functionality awareness is important to accomplish new tasks as well as achieving efficient completion of existing tasks. Systems to recommend commands have been used successfully to improve functionality awareness in large and complex software applications~\cite{matejka2009communitycommands,murphy2012improving}. Recommender systems are used generally to produce a list of predictions based on the preference of an user and the similarity of preferences of other users. In recent years, recommender systems have become quite popular and have been applied successfully in multiple sectors of business~\cite{linden2003amazon} and academia~\cite{hsu2008personalized,mcnee2006don}. 

% Our contribution.
Our contribution in this paper -- we recommended functions in spreadsheets by applying two distinct variations of collaborative filtering algorithm and compare the effectiveness of the recommendations using a cross validation based automated evaluation. We also compared their performance for recommending functions against another algorithm priorly used for recommending commands in large software systems, the most popular algorithm~\cite{linton2000owl}. And we conclude by discussing how the results of our work can be used to create an Microsoft Excel plugin that can assist users to increase functional awareness by recommending functions they can use.

%% \subsection{Double-blind reviewing}

%% This year we are doing double-blind reviewing: the reviewers will not know 
%% who the authors of the paper are. For submission, the NIPS style file will 
%% automatically anonymize the author list at the beginning of the paper.

%% Please write your paper in such a way to preserve anonymity. Refer to
%% previous work by the author(s) in the third person, rather than first
%% person. Do not provide Web links to supporting material at an identifiable
%% web site.

%%\subsection{Electronic submission}
%%
%% \textbf{THE SUBMISSION DEADLINE IS June 5, 2015. SUBMISSIONS MUST BE LOGGED BY
%% 23:00, June 5, 2015, UNIVERSAL TIME}

%% You must enter your submission in the electronic submission form available at
%% the NIPS website listed above. You will be asked to enter paper title, name of
%% all authors, keyword(s), and data about the contact
%% author (name, full address, telephone, fax, and email). You will need to
%% upload an electronic (postscript or pdf) version of your paper.

%% You can upload more than one version of your paper, until the
%% submission deadline. We strongly recommended uploading your paper in
%% advance of the deadline, so you can avoid last-minute server congestion.
%%
%% Note that your submission is only valid if you get an e-mail
%% confirmation from the server. If you do not get such an e-mail, please
%% try uploading again. 

%% \subsection{Keywords for paper submission}
%% Your NIPS paper can be submitted with any of the following keywords (more than one keyword is possible for each paper):

%% \begin{verbatim}
%% Bioinformatics
%% Biological Vision
%% Brain Imaging and Brain Computer Interfacing
%% Clustering
%% Cognitive Science
%% Control and Reinforcement Learning
%% Dimensionality Reduction and Manifolds
%% Feature Selection
%% Gaussian Processes
%% Graphical Models
%% Hardware Technologies
%% Kernels
%% Learning Theory
%% Machine Vision
%% Margins and Boosting
%% Neural Networks
%% Neuroscience
%% Other Algorithms and Architectures
%% Other Applications
%% Semi-supervised Learning
%% Speech and Signal Processing
%% Text and Language Applications

%% \end{verbatim}

\section{Related Work}
Researchers have been studying and analyzing spreadsheets to better comprehend the activity of the users and design better tools to assist them. Previous research has focused on extracting structured domain information from spreadsheets~\cite{hermans2010automatically} and visualizing spreadsheet using dataflow diagrams~\cite{hermans2011breviz} to enhance understandability of spreadsheets. For improving the maintainability and quality of the spreadsheet formulas, systems have been implemented to detect code smells in formulas and refactor them to resolve the detected smells~\cite{hermans2014detecting}. In contrast, our work attempts to increase the functionality awareness in spreadsheets by suggesting functions.

Moreover, some research has used recommender systems to help users of a large software system with vast set of functionality to learn these functionalities. One of the early notable attempts is the OWL system~\cite{linton2000owl} developed by Linton et al., which suggests commands to Microsoft Word users based on the most popular algorithm. In OWL, commands are recommended to an individual if she is not using certain commands at all but the community is using them on a highly frequent basis. The underlying assumption of this system is that all users of a software system have a similar command usage distribution. Recommender systems were developed to improve upon the system delineated by Linton.

Matejka and colleagues developed a collaborative filtering-based approach to recommend commands in AutoCAD, a computer aided drafting software, called CommunityCommands~\cite{matejka2009communitycommands}. Similarly based on the users' usage history, Murphy-Hill and colleagues studied several existing recommendation algorithms to recommend commands in Eclipse. The approach that we propose in this paper,  focuses on the algorithm used by OWL as a baseline and the user-based and item-based variations of the collaborative filtering algorithm used by CommunityCommands.

Both of the algorithms used in our work requires a source of function usage preference for the spreadsheet users' community. Although there are existent spreadsheet corpora like Enron~\cite{hermans2014enron} and EUSES~\cite{fisher2005euses} which have been valuable sources for spreadsheet analysts and researchers alike, in this paper we look into a very recently extracted spreadsheet corpus, Fuse~\cite{barik2015fuse}. Unlike Enron and EUSES which are very domain specific, Fuse contains a more diverse and much larger body of spreadsheets. Fuse is also reproducible, thus the recommender system in our work can leverage of a larger spreadsheet corpus if needed, which can be extracted by the system developed by Fuse's creators.

\section{Headings: first level}
\label{headings}

First level headings are lower case (except for first word and proper nouns),
flush left, bold and in point size 12. One line space before the first level
heading and 1/2~line space after the first level heading.

\subsection{Headings: second level}

Second level headings are lower case (except for first word and proper nouns),
flush left, bold and in point size 10. One line space before the second level
heading and 1/2~line space after the second level heading.

\subsubsection{Headings: third level}

Third level headings are lower case (except for first word and proper nouns),
flush left, bold and in point size 10. One line space before the third level
heading and 1/2~line space after the third level heading.

\section{Citations, figures, tables, references}
\label{others}

These instructions apply to everyone, regardless of the formatter being used.

\subsection{Citations within the text}

Citations within the text should be numbered consecutively. The corresponding
number is to appear enclosed in square brackets, such as [1] or [2]-[5]. The
corresponding references are to be listed in the same order at the end of the
paper, in the \textbf{References} section. (Note: the standard
\textsc{Bib\TeX} style \texttt{unsrt} produces this.) As to the format of the
references themselves, any style is acceptable as long as it is used
consistently.

As submission is double blind, refer to your own published work in the 
third person. That is, use ``In the previous work of Jones et al.\ [4]'',
not ``In our previous work [4]''. If you cite your other papers that
are not widely available (e.g.\ a journal paper under review), use
anonymous author names in the citation, e.g.\ an author of the
form ``A.\ Anonymous''. 


\subsection{Footnotes}

Indicate footnotes with a number\footnote{Sample of the first footnote} in the
text. Place the footnotes at the bottom of the page on which they appear.
Precede the footnote with a horizontal rule of 2~inches
(12~picas).\footnote{Sample of the second footnote}

\subsection{Figures}

All artwork must be neat, clean, and legible. Lines should be dark
enough for purposes of reproduction; art work should not be
hand-drawn. The figure number and caption always appear after the
figure. Place one line space before the figure caption, and one line
space after the figure. The figure caption is lower case (except for
first word and proper nouns); figures are numbered consecutively.

Make sure the figure caption does not get separated from the figure.
Leave sufficient space to avoid splitting the figure and figure caption.

You may use color figures. 
However, it is best for the
figure captions and the paper body to make sense if the paper is printed
either in black/white or in color.
\begin{figure}[h]
\begin{center}
%\framebox[4.0in]{$\;$}
\fbox{\rule[-.5cm]{0cm}{4cm} \rule[-.5cm]{4cm}{0cm}}
\end{center}
\caption{Sample figure caption.}
\end{figure}

\subsection{Tables}

All tables must be centered, neat, clean and legible. Do not use hand-drawn
tables. The table number and title always appear before the table. See
Table~\ref{sample-table}.

Place one line space before the table title, one line space after the table
title, and one line space after the table. The table title must be lower case
(except for first word and proper nouns); tables are numbered consecutively.

\begin{table}[t]
\caption{Sample table title}
\label{sample-table}
\begin{center}
\begin{tabular}{ll}
\multicolumn{1}{c}{\bf PART}  &\multicolumn{1}{c}{\bf DESCRIPTION}
\\ \hline \\
Dendrite         &Input terminal \\
Axon             &Output terminal \\
Soma             &Cell body (contains cell nucleus) \\
\end{tabular}
\end{center}
\end{table}

\section{Final instructions}
Do not change any aspects of the formatting parameters in the style files.
In particular, do not modify the width or length of the rectangle the text
should fit into, and do not change font sizes (except perhaps in the
\textbf{References} section; see below). Please note that pages should be
numbered.

\section{Preparing PostScript or PDF files}

Please prepare PostScript or PDF files with paper size ``US Letter'', and
not, for example, ``A4''. The -t
letter option on dvips will produce US Letter files.

Fonts were the main cause of problems in the past years. Your PDF file must
only contain Type 1 or Embedded TrueType fonts. Here are a few instructions
to achieve this.

\begin{itemize}

\item You can check which fonts a PDF files uses.  In Acrobat Reader,
select the menu Files$>$Document Properties$>$Fonts and select Show All Fonts. You can
also use the program \verb+pdffonts+ which comes with \verb+xpdf+ and is
available out-of-the-box on most Linux machines.

\item The IEEE has recommendations for generating PDF files whose fonts
are also acceptable for NIPS. Please see
\url{http://www.emfield.org/icuwb2010/downloads/IEEE-PDF-SpecV32.pdf}

\item LaTeX users:

\begin{itemize}

\item Consider directly generating PDF files using \verb+pdflatex+
(especially if you are a MiKTeX user). 
PDF figures must be substituted for EPS figures, however.

\item Otherwise, please generate your PostScript and PDF files with the following commands:
\begin{verbatim} 
dvips mypaper.dvi -t letter -Ppdf -G0 -o mypaper.ps
ps2pdf mypaper.ps mypaper.pdf
\end{verbatim}

Check that the PDF files only contains Type 1 fonts. 
%For the final version, please send us both the Postscript file and
%the PDF file. 

\item xfig "patterned" shapes are implemented with 
bitmap fonts.  Use "solid" shapes instead. 
\item The \verb+\bbold+ package almost always uses bitmap
fonts.  You can try the equivalent AMS Fonts with command
\begin{verbatim}
\usepackage[psamsfonts]{amssymb}
\end{verbatim}
 or use the following workaround for reals, natural and complex: 
\begin{verbatim}
\newcommand{\RR}{I\!\!R} %real numbers
\newcommand{\Nat}{I\!\!N} %natural numbers 
\newcommand{\CC}{I\!\!\!\!C} %complex numbers
\end{verbatim}

\item Sometimes the problematic fonts are used in figures
included in LaTeX files. The ghostscript program \verb+eps2eps+ is the simplest
way to clean such figures. For black and white figures, slightly better
results can be achieved with program \verb+potrace+.
\end{itemize}
\item MSWord and Windows users (via PDF file):
\begin{itemize}
\item Install the Microsoft Save as PDF Office 2007 Add-in from
\url{http://www.microsoft.com/downloads/details.aspx?displaylang=en\&familyid=4d951911-3e7e-4ae6-b059-a2e79ed87041}
\item Select ``Save or Publish to PDF'' from the Office or File menu
\end{itemize}
\item MSWord and Mac OS X users (via PDF file):
\begin{itemize}
\item From the print menu, click the PDF drop-down box, and select ``Save
as PDF...''
\end{itemize}
\item MSWord and Windows users (via PS file):
\begin{itemize}
\item To create a new printer
on your computer, install the AdobePS printer driver and the Adobe Distiller PPD file from
\url{http://www.adobe.com/support/downloads/detail.jsp?ftpID=204} {\it Note:} You must reboot your PC after installing the
AdobePS driver for it to take effect.
\item To produce the ps file, select ``Print'' from the MS app, choose
the installed AdobePS printer, click on ``Properties'', click on ``Advanced.''
\item Set ``TrueType Font'' to be ``Download as Softfont''
\item Open the ``PostScript Options'' folder
\item Select ``PostScript Output Option'' to be ``Optimize for Portability''
\item Select ``TrueType Font Download Option'' to be ``Outline''
\item Select ``Send PostScript Error Handler'' to be ``No''
\item Click ``OK'' three times, print your file.
\item Now, use Adobe Acrobat Distiller or ps2pdf to create a PDF file from
the PS file. In Acrobat, check the option ``Embed all fonts'' if
applicable.
\end{itemize}

\end{itemize}
If your file contains Type 3 fonts or non embedded TrueType fonts, we will
ask you to fix it. 

\subsection{Margins in LaTeX}
 
Most of the margin problems come from figures positioned by hand using
\verb+\special+ or other commands. We suggest using the command
\verb+\includegraphics+
from the graphicx package. Always specify the figure width as a multiple of
the line width as in the example below using .eps graphics
\begin{verbatim}
   \usepackage[dvips]{graphicx} ... 
   \includegraphics[width=0.8\linewidth]{myfile.eps} 
\end{verbatim}
or % Apr 2009 addition
\begin{verbatim}
   \usepackage[pdftex]{graphicx} ... 
   \includegraphics[width=0.8\linewidth]{myfile.pdf} 
\end{verbatim}
for .pdf graphics. 
See section 4.4 in the graphics bundle documentation (\url{http://www.ctan.org/tex-archive/macros/latex/required/graphics/grfguide.ps}) 
 
A number of width problems arise when LaTeX cannot properly hyphenate a
line. Please give LaTeX hyphenation hints using the \verb+\-+ command.


\subsubsection*{Acknowledgments}

Use unnumbered third level headings for the acknowledgments. All
acknowledgments go at the end of the paper. Do not include 
acknowledgments in the anonymized submission, only in the 
final paper.

\bibliographystyle{unsrt}
\bibliography{report}

\end{document}
